\section{Метод решения}

Для решения задачи применена архитектура с двумя процессами (родительским и дочерним), взаимодействующими через два неименованных канала (pipe). 

\subsection{Основной алгоритм работы}
\begin{enumerate}
    \item \textbf{Инициализация:} Создание двух каналов - pipe1 для передачи данных от родителя к потомку, pipe2 для обратной связи (сообщения об ошибках)
    \item \textbf{Запуск процесса:} Создание дочернего процесса с перенаправлением стандартных потоков ввода/вывода
    \item \textbf{Передача параметров:} Отправка имени файла через pipe1 как первого сообщения
    \item \textbf{Обработка данных:}
    \begin{itemize}
        \item Родительский процесс читает строки от пользователя и передает через pipe1
        \item Дочерний процесс проверяет каждую строку на соответствие критерию (начало с заглавной буквы)
        \item Валидные строки записываются в файл, сообщения об ошибках отправляются через pipe2
    \end{itemize}
    \item \textbf{Завершение работы:} Корректное закрытие каналов и процессов при получении пустой строки
\end{enumerate}

\subsection{Особенности реализации}
Для обеспечения кроссплатформенности разработан уровень абстракции, скрывающий различия между API Windows и Unix-систем. Реализована поддержка как латинских, так и кириллических символов при проверке заглавных букв.

\section{Описание программы}

Программа реализована в модульном стиле и состоит из четырех основных компонентов.

\subsection{Модуль parent.c}
Реализует логику родительского процесса:
\begin{itemize}
    \item Создание и управление каналами связи
    \item Запуск и контроль дочернего процесса
    \item Взаимодействие с пользователем (ввод строк)
    \item Координация передачи данных между процессами
    \item Обработка сообщений об ошибках от дочернего процесса
\end{itemize}

\subsection{Модуль child.c}
Содержит бизнес-логику дочернего процесса:
\begin{itemize}
    \item Чтение входных данных из канала pipe1
    \item Валидация строк по критерию (начало с заглавной буквы)
    \item Запись валидных строк в выходной файл
    \item Формирование и отправка сообщений об ошибках через pipe2
    \item Управление файловыми операциями
\end{itemize}

\subsection{Модуль cross\_platform.h/c}
Предоставляет кроссплатформенные абстракции:
\begin{itemize}
    \item \textbf{Структуры данных:} pipe\_t (для каналов), process\_t (для процессов)
    \item \textbf{Функции работы с каналами:} создание, закрытие, чтение, запись
    \item \textbf{Функции управления процессами:} создание, ожидание завершения
    \item \textbf{Вспомогательные функции:} перенаправление потоков, работа с памятью
\end{itemize}

\subsection{Модуль string\_utils.h/c}
Содержит функции обработки строк:
\begin{itemize}
    \item \texttt{is\_capital\_start()} - проверка начала строки с заглавной буквы с поддержкой латиницы и кириллицы
    \item \texttt{trim\_newline()} - удаление символов новой строки
\end{itemize}

\subsection{Используемые системные вызовы}
\begin{itemize}
    \item \textbf{Windows:} CreateProcess, CreatePipe, ReadFile, WriteFile, CloseHandle
    \item \textbf{Unix:} fork, pipe, dup2, read, write, close, waitpid
    \item \textbf{Кроссплатформенные:} fopen, fclose, fgets, fprintf, fflush
\end{itemize}

Архитектура программы обеспечивает четкое разделение ответственности между модулями и поддерживает работу в различных операционных средах.