\section{Условие}
Родительский процесс создает дочерний процесс. Первой строкой пользователь в консоль
родительского процесса вводит имя файла, которое будет использовано для открытия File с таким
именем на запись. Перенаправление стандартных потоков ввода-вывода показано на картинке
выше. Родительский и дочерний процесс должны быть представлены разными программами.
Родительский процесс принимает от пользователя строки произвольной длины и пересылает их в
pipe1. Процесс child проверяет строки на валидность правилу. Если строка соответствует правилу,
то она выводится в стандартный поток вывода дочернего процесса, иначе в pipe2 выводится
информация об ошибке. Родительский процесс полученные от child ошибки выводит в
стандартный поток вывода.

{\bfseries Цель работы:} Приобретение практических навыков управления процессами в операционных системах семейства Windows и Linux/Unix, а также организация межпроцессного взаимодействия с использованием неименованных каналов (pipes). Дополнительной целью являлась разработка кроссплатформенного решения, абстрагирующего особенности системных API.

{\bfseries Задание:} Разработать программу из трёх процессов: родительского и двух дочерних, взаимодействующих через неименованные каналы.

Родительский процесс должен:
\begin{itemize}
    \item Запрашивать у пользователя имя файла и передавать его дочерним процессам
    \item Принимать от пользователя строки и распределять их по чётности:
    \begin{itemize}
        \item Нечётные строки (по порядку ввода) отправлять в pipe1 первому дочернему процессу
        \item Чётные строки отправлять в pipe2 второму дочернему процессу
    \end{itemize}
\end{itemize}

Дочерние процессы должны:
\begin{itemize}
    \item Получать строки из соответствующих каналов
    \item Инвертировать полученные строки (записывать символы в обратном порядке)
    \item Записывать результат в общий выходной файл
\end{itemize}

{\bfseries Вариант:} 15


