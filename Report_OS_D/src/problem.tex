\section{Условие}
Родительский процесс создает дочерний процесс. Первой строкой пользователь в консоль
родительского процесса вводит имя файла, которое будет использовано для открытия File с таким
именем на запись. Перенаправление стандартных потоков ввода-вывода показано на картинке
выше. Родительский и дочерний процесс должны быть представлены разными программами.
Родительский процесс принимает от пользователя строки произвольной длины и пересылает их в
pipe1. Процесс child проверяет строки на валидность правилу. Если строка соответствует правилу,
то она выводится в стандартный поток вывода дочернего процесса, иначе в pipe2 выводится
информация об ошибке. Родительский процесс полученные от child ошибки выводит в
стандартный поток вывода.

{\bfseries Цель работы:} Приобретение практических навыков управления процессами в операционных системах семейства Windows и Linux/Unix, а также организация межпроцессного взаимодействия с использованием неименованных каналов (pipes). Дополнительной целью являлась разработка кроссплатформенного решения, абстрагирующего особенности системных API.

{\bfseries Задание:} Разработать программу, состоящую из двух процессов — родительского и дочернего, взаимодействующих через неименованные каналы.

Родительский процесс должен:
\begin{itemize}
    \item Запрашивать у пользователя имя файла и передавать его дочернему процессу;
    \item Принимать от пользователя строки и передавать их дочернему процессу;
    \item Получать от дочернего процесса сообщения о результатах обработки строк и выводить их на экран.
\end{itemize}

Дочерний процесс должен:
\begin{itemize}
    \item Получить от родительского процесса имя файла и открыть его для записи;
    \item Принимать строки от родительского процесса;
    \item Проверять, начинается ли каждая строка с заглавной буквы;
    \item Если строка начинается с заглавной буквы — записывать её в файл;
    \item Если строка не начинается с заглавной буквы — отправлять сообщение об ошибке родительскому процессу;
    \item Завершать работу после получения пустой строки.
\end{itemize}

{\bfseries Вариант:} 15


