\section{Выводы}

В ходе выполнения лабораторной работы были успешно достигнуты все поставленные цели и решены основные задачи:

\begin{enumerate}
    \item \textbf{Освоены механизмы управления процессами:} На практике применены системные вызовы для создания и управления процессами в различных операционных системах (\texttt{fork()}, \texttt{waitpid()} в Unix и \texttt{CreateProcess()}, \texttt{WaitForSingleObject()} в Windows)
    
    \item \textbf{Реализовано межпроцессное взаимодействие:} Организован эффективный обмен данными между независимыми процессами с использованием неименованных каналов (pipe), что позволило обеспечить разделение функциональности между родительским и дочерним процессами
    
    \item \textbf{Создано кроссплатформенное решение:} Разработана система абстракций, позволяющая программе компилироваться и работать в различных операционных системах без изменения бизнес-логики приложений
    
    \item \textbf{Решены практические проблемы:} 
    \begin{itemize}
        \item Реализована поддержка многобайтовых кодировок (UTF-8) для корректной обработки кириллических символов
        \item Организовано асинхронное чтение из каналов для своевременного получения сообщений об ошибках
        \item Обеспечено корректное освобождение системных ресурсов (дескрипторов файлов и процессов)
    \end{itemize}
\end{enumerate}

Работа продемонстрировала важность создания переносимого и устойчивого к ошибкам программного обеспечения, а также необходимость тщательного проектирования архитектуры приложений, использующих межпроцессное взаимодействие. Полученные навыки могут быть применены при разработке более сложных распределенных систем и приложений, требующих параллельной обработки данных.

\pagebreak