\section{Результаты}

В результате работы была разработана кроссплатформенная программа для межпроцессного взаимодействия, успешно функционирующая как в Windows, так и в Unix-подобных операционных системах.

\subsection{Ключевые особенности реализации}

\begin{itemize}
    \item \textbf{Кроссплатформенная архитектура:} Программа использует единый код для различных ОС благодаря системе абстракций в модуле \texttt{cross\_platform}
    \item \textbf{Поддержка Unicode:} Реализована проверка заглавных букв как для латинского алфавита (ASCII), так и для кириллицы (UTF-8)
    \item \textbf{Асинхронная обработка ошибок:} Родительский процесс проверяет канал ошибок без блокировки основного потока выполнения
    \item \textbf{Корректное управление ресурсами:} Обеспечено правильное закрытие дескрипторов каналов и процессов при завершении работы
\end{itemize}

\subsection{Пример работы программы}

\begin{verbatim}
Enter file name: output.txt
Enter string (empty string to exit): hello world
Child: Error: string must start with capital letter - 'hello world'
Enter string (empty string to exit): MAI
Enter string (empty string to exit): 123Start
Child: Error: string must start with capital letter - '123Start'
Enter string (empty string to exit): Aviation
Enter string (empty string to exit):
Parent process finished.
\end{verbatim}

\noindent\textbf{Содержимое файла output.txt:}
\begin{verbatim}
MAI
Aviation
\end{verbatim}

\subsection{Производительность}

Программа демонстрирует стабильную работу при обработке строк различной длины. Время отклика системы на ввод пользователя практически не отличается от времени работы обычных консольных приложений, что подтверждает эффективность выбранного подхода к организации межпроцессного взаимодействия.
